\chapter{Abstract}

Nowadays the amount of available digital information is exponentially increasing. However, to use that raw data it is necessary to process it with some method. One of those methods is clustering, which is used to categorize data effectively. Thanks to it we can reduce the size of large data sets by extracting the most relevant information, usually the common features of a group or a subset of representatives. With the aforementioned dramatical increase in data size, researchers have focused on improving performance as much as possible. The cost of managing and storing information has considerably dropped rendering modern supercomputers an accessible asset to speed up clustering tools. On this scenario, we find pyProCT clustering tool and pyCOMPSs parallelization framework. The first is a software originally designed to work with proteins, hence the name \textbf{Py}thon \textbf{Pro}tein \textbf{C}lustering \textbf{T}ool, which has evolved into a general purpose clustering analysis program. The later is a framework built at the Barcelona Supercomputing Center that facilitates the development of parallel computational workflows in Python. It is focused on providing an easy programming model able to run on supercomputers and cloud resources. This project will refactor pyProCT with the pyCOMPSs framework to increase its performance by optimizing its executions on MareNostrum III supercomputer.

La quantitat d'informació digital disponible s'ha vist incrementada exponencialment ens els últims anys. Emperò, abans d'usar aquestes dades s'han de processar d'alguna manera. Un dels métodes utilitzats per a aquesta fi és el clustering, que es basa en categoritzar les dades efectivament. Gràcies al seu ús és possible reduir grans conjunts de dades mitjançant l'extracció de la informació més rellevant, habitualment els trets distintius d'un grup o subconjunt de elements representants. Tenint en compte el gran augment del tamany de les dades mençionat abans, les línies de recerca solen estar orientades en incrementar al màxim el rendiment. El cost de gestionar i emmagatzemar informació ha disminuït notablement últimament. Això ha fet que els superodinadors siguin accessibles  a un major públic i convertint-los en un gran recurs per millorar les eines de clustering. És en aquest escenari que es presenten l'eina de clustering pyProCT i l'entorn de treball pyCOMPSs. El primer és un programa creat originalment per a treballar amb proteines que ha evolucionat fins a convertir-se una eina de clustering de propòsit general. El segon, per la seva banda, és un entorn de treball, creat al centre de supercomputació de Barcelona, que facilita el desenvolupament de d'aplicacions paral·leles en Python. Està enfocat a proporcionar una interfície d'ús senzilla mentre a la vegada que la capacitat de ser executat en superodinadors i recursos en el núvol. L'objectiu d'aquest projecte és modificar pyProCT per a que usi pyCOMPSs. Esperem que això millori el seu rendiment gràcies a l'execució optimizada per a superodinadors com MareNostrum III, on es faran les proves.
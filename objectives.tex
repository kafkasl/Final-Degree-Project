\chapter{Objectives}

The goal of this project is to refactor pyProCT with pyCOMPSs programming model and framework. In order to achieve that three objectives were defined:

\begin{itemize}
\item \textbf{Understand pyProCT} 
\item \textbf{Refactor pyProCT} with pyCOMPSs
\item \textbf{Validate the results}
\end{itemize}


\section{Understand pyProCT}

The first objective was to understand and analyse pyProCT. A complete description of the task's scheduling and pipeline was necessary to decide how to refactor it with pyCOMPSs. The methodology section \ref{sec:sdd} contains the software design description with detailed information of each section of the program and its control parameters.

\section {Refactor pyProCT with pyCOMPSs}

This objective comprises the code adaptation to make pyProCT work with the pyCOMPSs framework. The section \ref{sec:refactor} is composed of two parts: the initial setup required to test the code under development and the actual refactor with pyCOMPSs. The analysis tools research and a more detailed description of the old pyProCT scheduler can be found on appendix \ref{chap:ttt}.

\section{Validate the results}
 
The last objective refers to the correctness of the refactored code. We need to ensure that the new scheduler produces the same results as the old ones. To do so we used the validation software called pyProCT which will be described in section \ref{sec:regression}.
\section{Refactor}
\label{sec:refactor}

This section will walk you through all the refactor process. It will provide a full description of the issues found, wether they were solved or not, the design decisions made and the reasons behind them, and all the information relevant for debugging, testing and further developing both pyCOMPSs and pyProCT. 


\subsection{Set up}


The installation of pyProCT as described on the \hyperref[sec:docs]{pyProCT repo} is trivial on a local machine. On MareNostrum III pyProCT is already installed.  In order to use my version under development (instead of the package installed both on MN3 or a local machine) the user just needs to point the python path to it. This is useful to switch between different working versions (for example to use pyProCT-regression validation or to meet the different instrumentation requirements of each scheduler). Later some issues will also force me to use this same method to customize some of the dependencies of pyProCT such as the pyScheduler or pyRMSD. 

Once I installed and ran a few pyProCT examples on my local machine I proceeded to MareNostrum III to do the same. Choosing a good structure to set up the environment is a must for executions on MN3. 

I faced and spent a lot of time on configuration problems. This kind of issues kept popping up during the whole project. Despite that I prefered to group them here and give a brief description of the issues and how they were solved.

This first issue arose when trying to compile and link the development version of pyProCT. It is related with MN3's modules environment. By default, on login, MN3 has 2.6.9 python, however this version is not available to be loaded through the modules; it's only available when no other python has been loaded by the .bashrc file nor manually with \textit{module load PYTHON}. 

pyProCT depends on python 2.7.3 which can be loaded with the modules. At first I compiled and installed it with the default release (2.6.9) with setup.py. On MN3 I had to add a custom installation path (with \textit{--prefix=PATH} option) to setup.py because I have no permissions to write into the default installation directory. After installing it I found out that pyProCT can not be run under python 2.6.9 so I started again all the installation process with 2.7.3 once I figured what was causing the error. 

The new installation lead to the following new bug: 

\textit{undefined symbol: PyUnicodeUCS4\_DecodeUTF8}

This is caused when trying to use software build with UCS4 on a UCS2 python version. On MN3 each installation uses a different one.

\begin{itemize}
\itemsep0em 
\item Python 2.7.3 $\rightarrow$ UCS2
\item Python 2.6.9 $\rightarrow$ UCS4
\end{itemize}

Python is not a compiled language, so this compilation problem actually comes from the Cython modules integrated into pyProCT. This meant that the new installation (which used the same folder as source) was not recompiling the Cython modules even after issuing a clean command so I cloned the repo again and started from scratch. This time everything ran smoothly. As a curiosity if pyProCT is build with python2.6.9 it can be used with python2.7.3 (although rising some compatibility warnings).

\subsection{pyCOMPSs}
\label{subsec:pycompss}

Prior to starting the refactor I analyzed which would be the best way to parallelize it. pyCOMPSs works by using python's decorators to define some functions as \textit{COMPSs' tasks}. These tasks are executed on previously defined resources such as a MN3 node or a cloud. For each task the framework checks whether that function's parameters depend on some previous task; if it has no dependencies then the task is assigned to a resource which runs it. 

pyProcT clustering and postprocessing sections, as previously stated, are embarrassingly parallel: all algorithm's executions depend only on the distance's matrix calculation; the postprocessing actions all depend on the best clustering (that is to say: the whole clustering section). Knowing this we decided to define as task each algorithm execution and each postprocessing action.

I wanted to maintain the possibility to use the other schedulers after the refactor so I kept the overall structure of pyProCT. However, I also wanted to exploit the possibility of reduce the code complexity while achieving the maximum performance improvement. I mention this because make the sequential version of pyProCT work with pyCOMPSs is enough to place the decorators on the right functions. It is true that this would also raise some issues to be addressed; my point is that the lines of code required are few if the goal is just to make it work. This is not the goal though. The refactor described from here onwards tries to minimize the code size, make it clearer. It also removes functionality duplication between the framework and the software. For example pyProCt has a loop which enqueues the tasks for the scheduler. COMPSs also has an internal queuing system rendering this loop unnecessary.

Bearing this in mind, differences are basically found on the Driver, Protocol and Explorer classes, which deal respectively with all the sections pipeline execution, the clustering pipeline, and the clustering exploration \textit{per se}. I simply created a new Driver for the COMPSs scheduling. The main checks whether pyCOMPSs is the scheduler or not and calls one driver or the other accordingly (same method being used for MPI). From the driver onwards the key classes are substituted by the COMPSs versions.

One of the advantages of pyCOMPSs is the small amount of work required to use it. On a normal sequential program we just need to use the \textit{@task()} decorator and the \textit{obj = compss\_wait\_on(obj)} API call to create synchronization points for future objects; from \hyperref[sec:docs]{COMPSs manual}: 

\begin{quote} 
If the programmer defines, as a task, a function or method that returns a value, that value is not generated until the task is executed. However, in order to keep the asynchrony of the task invocation, COMPSs manages future objects: a representant object is immediately returned to the main program when a task is invoked.
\end{quote}

Internally COMPSs has queue of tasks so the step to add the tasks to the scheduler is no longer required; instead I called directly the decorated methods (which internally COMPSs enqueues to it's pending's list). However this caused a problem related to the how the framework deals with the data.

COMPSs is a framework which allows to define a lot of different resources. The communication layer needs a high level of abstraction because workers (resources able to execute tasks) use different protocols (e.g. SSH or NIO). To send the data needed by each task, that is, the method's parameters, COMPSs serializes them to Java objects (except basic types). This means that python's pickle must be able to do the translation which is not the case for the distances matrix.

PyProCT uses pyRMSD to represent the forementioned matrix. It is basically a python wrapper for a C matrix structure. The goal of this implementation is to highly reduce the access time to the matrix elements. 
\begin{quote}
Python is slow [...], why: it boils down to Python being a dynamically typed, interpreted language, where values are stored not in dense buffers but in scattered objects. \cite{vanderplas_why_2014}
\end{quote}

To overcome this, Víctor wrote the lean and specialized pyRMSD (which stands for python Root Mean Squared Deviation). The problem is that these structure is not a native python type nor it's built with a combination of them. Because of this the framework can not serialize this matrix to send it to each worker. The first idea to solve it was to dump the internal data of the matrix into a python list (which can be serialized arbitrarily) with the already implemented method \textit{get\_data}; this list then can be passed to the class constructor \textit{CondensedMatrix()} obtaining again the original one. This approach raised two issues. 

The first issue was where to perform the translations to a python list, which is linked to which is the function decorated as task. Before presenting the solution and the next issue it is worth to make a point about how the algorithms are implemented and called. 

Each algorithm is implemented in a different class so we have \textit{kMedoidsAlgorithm.py}, \textit{spectralClusteringlAlgorithm.py} and so on. All of them however return the same kind of data: clusterings; in order to simplify all the execution pipeline and the code they all have the same structure: all the required arguments are passed to the class constructor and then they all have the \textit{perform\_clustering} method which returns the clusterings found for that parametrization. 

With this structure then it would be necessary to decorate the \textit{perform\_clustering} method of each algorithm but only when pyCOMPSs scheduling is used. In order to avoid having code duplication (one with the decorator and one without for each algorithm) I implemented a wrapper class, called CompssTask, for the algorithm's execution with a single pyCOMPSS-decorated method.

The class is constructed with all the required information for the task execution. During the initialization I also do the forementioned translation of the matrix to a python list. After the constructor the run method, which is the actual pyCOMPSs task executed on a worker, recreates the Condensed Matrix from the python list and executes the algorithm's clustering method. 

After the execution of the algorithm it was necessary to again deassign the computed matrix because it is part of the class and, even if it's not a result, when the task is finished pyCOMPSs again tries to serialize the object and fails.

Finally I will talk about the hardware abstraction of COMPSs. Thanks to it the user does not need to worry a

\subsection{Other issues}


When trying to generate traces with extrae (for MPI and sequential version) I got an Prody error. When trying to resize any kind of structure Prody detects that there is more than one reference to that structure (introduced by the instrumentation) and fails to do the resize. To avoid this I had to manually modify the Prody package and, for each resize, add the parameter \textit{refcheck=False}. This error is raised in order to avoid integrity problems when an object has more than one reference; however we know that the instrumentation will not modify nor actively use those structures so we can safely disable the reference's check. 

Another issue was raised by some datasets. Depending on the computer and data when I try to recreate the condensed matrix on the pyCOMPSs task I get an incompatible format error. This happens when numpy stores the matrix data (in list format) as floats with 32 bits. The matrix constructor however requires that data to be in floats with 64 bits. To overcome this I found that numpy arrays have a method, \textit{data\_view('float64')}, to select which type of elements should be returned and thus allowing me to always format them as 64-bit floats and solve the issue.










\subsection{Other issues}


When trying to generate traces with extrae (for MPI and sequential version) I got an Prody error. When trying to resize any kind of structure Prody detects that there is more than one reference to that structure (introduced by the instrumentation) and fails to do the resize. To avoid this I had to manually modify the Prody package and, for each resize, add the parameter \textit{refcheck=False}. This error is raised in order to avoid integrity problems when an object has more than one reference; however we know that the instrumentation will not modify nor actively use those structures so we can safely disable the reference's check. 

Another issue was raised by some datasets. Depending on the computer and data when I try to recreate the condensed matrix on the pyCOMPSs task I get an incompatible format error. This happens when numpy stores the matrix data (in list format) as floats with 32 bits. The matrix constructor however requires that data to be in floats with 64 bits. To overcome this I found that numpy arrays have a method, \textit{data\_view('float64')}, to select which type of elements should be returned and thus allowing me to always format them as 64-bit floats and solve the issue.








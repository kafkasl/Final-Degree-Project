\chapter{Budget}

This section describes the required budget for the pyProCT optimization project. It contains a detailed description of the material and human costs. The description is divided into: human, hardware and software resources; instead of specifying the costs per tasks, we have decided to use this structure because there are not remarkable differences on the resources used for each task so grouping them like this the document will be clearer, will avoid too many subsections and, on the Temporal Planning, the resources needed for each task have already been specified.

\section{Human Resources}


This project will be designed, developed and evaluated by one person. However I divide it into two roles: one will correspond with the work on the actual project, the other is going to represent the time spent on developing COMPSs as well as the questions and issues. The estimation of the human cost is tied to the work time represented, in this case, by the Gantt Chart and the task description provided on the \hyperref[sec:gantt_pert]{\ref*{sec:gantt_pert} Gantt and PERT charts} section of the temporal planning. 

The project will be completed single-handedly developer with an eight-long workday from Monday to Friday, without having holidays.

First is important to note that the Project Management task overlaps with other tasks. However the duration of this section is tied to the programmed schedule of the GEP course, not to the amount of work required to finish it. Thanks to that, we will consider that from 12th February to 12th March the workday is going to be equally distributed amongst the overlapping tasks, which are less work-intensive because they are merely familiarization and research tasks. Similarly, because the results are tied to the development of COMPSs new release, I will start writing the report before having all the results. This way I will have almost everything ready before having the results. 

Taken that into account, the defined work period has 176 workdays from 12th February to 13th October (12d + 22d + 22d + 22d + 5d monthly breakdown) amounting to approximately a total of 1400 hours. So the human cost will be:
\\

\begin{center}
	\begin{tabular}{| R{3cm} | L{3cm} | L{3cm} | L{3cm} |}
	\hline
	Role & Price per hour & Time & Cost \\ 
	\hline
	\hline
	Project Developer & 10,00 \euro & 700h & 6.640,00 \euro \\
	\hline
	BSC Developer & 10,00 \euro & 700h & 6.640,00 \euro \\
	\hline
	\end{tabular}
	\captionof{table}{Human Resources Budget} 
\end{center}


\section{Hardware Resources}
\label{sec:hardware_resources}
The hardware resources for this software project are going to be the development device, a laptop, and the testing one, the Mare Nostrum III. It's assumed that the computer used for development has an internet connection and electrical connection. These costs are covered on the total budget together with unexpected costs. However to reduce the budget one possibility would be to consider using the university facilities. The university provides to it's students and developers a free network and plugs which is more than enough in this case.


\makesavenoteenv{tabular}
\begin{center}
	\begin{tabular}{| R{4cm} | L{3cm} | L{2cm} | L{3cm} |}
	\hline
	Product & Price & Useful life & Amortisation \\ 
	\hline \hline
	Mare Nostrum III & 22.700.000,00 \euro & 3 years & 0\protect\footnote{ MareNostrum III is a public infrastructure so users need not to pay to use it} \euro \\
	\hline
	Laptop & 1.200,00 \euro & 3 years & 150,09\footnote{ Given by: Cost / Useful life * Time used on project (664h)} \euro \\
	\hline
	\hline
	Total & 22.701.200,00 \euro & & 150,09 \euro \\
	\hline
	\end{tabular}
	\captionof{table}{Hardware Resources Budget}
\end{center}


\section{Software Resources}
\label{sec:software_resources}

pyProCT is an open source software hosted on a public github repository which can be used without restriction subject to the condition of citing the following article:

Copyright (C) 2012 Víctor Alejandro Gil Sepúlveda
pyProCT: Automated Cluster Analysis for Structural Bioinformatics
J. Chem. Theory Comput., 2014, 10 (8), pp 3236–3243
DOI: 10.1021/ct500306s \\
As our aim is to improve this software we want to keep it as it is. This means, on one hand, that all the features and optimizations added to it will also be free and public, using no third-party paying software. On the other hand, being it a public software we have decided that the development will allow reproducible research, meaning that all the tools used for analysis are also going to be free and available to anyone trying to reproduce the analysis and optimizations of this project.



\begin{center}
	\begin{tabular}{| R{4cm} | L{3cm} | L{2cm} | L{3cm} |}
	\hline
	Product & Price & Useful life & Amortisation \\ 
	\hline \hline
	Linux Mint 17.1 & 0,00 \euro & - & 0,00 \euro \\
	\hline
	Extrae & 0,00 \euro & - & 0,00 \euro \\
	\hline
	Paraver & 0,00 \euro & - & 0,00 \euro \\
	\hline
	Git & 0,00 \euro& - & 0,00 \euro \\
	\hline
	Github account\footnote{ The repository is public so no premium account is required}& 0,00 \euro & - & 0,00 \euro \\
	\hline
	Texstudio & 0,00 \euro & - & 0,00 \euro \\
	\hline
	GanttProject & 0,00 \euro & - & 0,00 \euro \\
	\hline
	Dia2code (UML drawing)& 0,00 \euro & - & 0,00 \euro \\
	\hline
	Atenea UPC& 0,00 \euro & - & 0,00 \euro \\
	\hline
	Other tools & 0,00 \euro & - & 0,00 \euro \\
	\hline
	\hline
	Total & 0,00 \euro \euro & & 0,00 \euro \euro \\
	\hline
	\end{tabular}
	\captionof{table}{Software Resources Budget}
\end{center}



\section{Total Budget}

Adding up all the cost described on the previous section we get total cost of the project, to which we need to add the VAT, which is 21 \% in Spain. We do not expect big problems or incidents because, as we stated, we aim to use only free software so any modification or change on the task's planning will mainly just add office rental costs (taking into account that the office rental also includes the electricity and internet costs).

To control unexpected events we will add to the Total Cost an amount of money to confront them. These would cover various problems such as: an electricity or internet cost rise, more required office time (rising the rental costs and network/electricity) or, in case of not having enough time, the hiring of supporting help (other developers).


\begin{center}
	\begin{tabular}{| R{4cm} | L{3cm} | L{2cm} | L{3cm} |}
	\hline
	Product & Price & Useful life & Amortisation \\ 
	\hline \hline
	MareNostrum III & 22.700.000,00 \euro & 3 years & 0,00\footnote{ MareNostrum III is a public infrastructure so users need not to pay to use it} \euro \\
	\hline
	Laptop & 1.200,00 \euro & 3 years & 150,09\footnote{ Given by: Cost / Useful life * Time used on project (664h)} \euro \\
	\hline
	Office rental & 5.000,00 \euro & - & 5.000,00 \euro \\
	\hline
	Unexpected costs & 3.000,00 \euro & - & 3.000,00\footnote{ Given by: Cost / Useful life * Time used on project (664h)} \euro \\
	\hline
	\hline
	Subtotal & 30.701.200,00 \euro & - & 8.150,09 \euro \\
	\hline
	VAT (21 \%) & 6.447.252,00 \euro & - & 1.711,519 \euro \\
	\hline
	\hline
	Total & 37.148.452,00 \euro & - & 9.861,609 \euro \\
	\hline
	\end{tabular}
	\captionof{table}{Total Budget}
\end{center}

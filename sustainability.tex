\chapter{Sustainability}

\section{Economic}

An economic assessment has already been described on the Budget Section. The resources used for the development have been kept to a minimum; trying to use all the free software and resources provided by the university, in which a project like this could be developed; being developed by a single person, implying just one salary. The time used, however, could be reduced by having a developer and an analyst. This way on the SCRUB iterations while the developer is working a new feature the analyst could be working on the results of the previous one and so on. A part from that it's difficult to reduce more the amount of work because we have already considered a full-length workday without holidays.

On the major optimization, the COMPSs refactor, we are utilising an already developed framework which eases a lot the amount of work required for it. Thanks to the collaboration with the COMPSs developing team we achieve, as said, a faster implementation, good support, because the framework is currently used, and also we help a good framework as COMPSs to gain more notoriousness and relevance on research projects.

This project will have an 8 in the economical viability area because the cost can hardly be reduced. However, performing a more in-depth ot the risks of the project could help to reduce the budget for unexpected problems.



\section{Social}

On the social dimension we find that the optimization of this software will allow more research teams or enterprises to use it. This is important because on large datasets the amount of time and processing power can be overwhelming for small teams. Even if an optimization as this can not directly change a whole country, it could help universities and developers to waste less time of Supercomputing centers which are quite expensive and so improve their performance and resource disposal and use. As said, these project will not produce better results for the end user but will help the HPC providers.

The optimization covers a necessity as this project is done on BSC demand. So they will benefit from it, harming no one else on the process.

On the social area it will also receive an 8 because it will help to improve and further develop the COMPSs framework on the industry, providing more data, cases of use and information to the BSC/COMPSs development team.

\section{Environmental}

The environmental-related resources of this project are, primarily, two: the Mare Nostrum III supercomputer and the development laptop. 

Both resources use electric energy. Theoretically this project will diminish the energy used by the supercomputer, which is quite high, but, in fact, the supercomputer is always running so the impact will be almost negligible, from a power-consumption point of view. However, on smaller scale devices this could in fact reduce a little the amount of energy spent, not worsening it in any case.

The COMPSs framework is used to help the development so we are reusing previously done work, giving the original project more scope and giving it more usage, helping to make the most out of the framework development.

This a software project so no other resources than the electricity used to run it is required. No direct waste is going to be produced by it's use. As it is an open source project aimed to be used or improved the whole project can be reused, both for new projects aiming to reduce even further the power and time consumption and for teams requiring a generic clustering analysis method.

On the resources analysis the project will be awarded with a 9 because the only thing that it's not environmental friendly is the electricity consumed by the Mare Nostrum III and the laptop to run the program and, compared to the average consumption per person nowadays, it's not a big deal.